% \section{Boolean Plus/Minus One Formulation}
% For a boolean variable parametrized by $\theta_k\in \{-1, 1\}$, the problem becomes
% \begin{mini*}|l|
%     {\mbs{\xi}, \mbs{\theta}}{\sum_{k=1}^K \frac{\theta_k+1}{2} r_k^2(\mbs{\xi})}{}{}\tag{\text{MIX 2}}
%     \label{prob:mix_boolean_pm_one}
%     \addConstraint{\theta_k}{\in \{-1, 1\}}
%     \addConstraint{\sum_{k=1}^K \theta_k}{=-K+2}
% \end{mini*}
% The lifted variables and the moment matrix remain the same as in the boolean zero or one case.
% However, the each summand of the objective in \ref{prob:mix_boolean_pm_one} becomes
% \begin{align}
%     S_k & = \frac{\theta_k+1}{2} r_k^2(\mbs{\xi}),
% \end{align}
% and must be formulated in terms of the lifted variables
% $\OneVariable ,\theta_k ,\theta_k \mbs{\xi}$ as
% \begin{align}
%     S_k & = \frac{\theta_k+1}{2} r_k^2(\mbs{\xi}) \\
%         & =
%     \frac{\theta_k+1}{2}(\mbsxi^\trans \mbf{Q}_{\mbsxi, k} \mbsxi
%     +
%     \mbf{b}^\trans \mbsxi
%     + c_\mbsxi)
% \end{align}