\section{Discretization}

Using a Zero-Order-Hold (ZOH) discretization, the continuous-time system is converted into 
the following discrete-time state-space form:

\begin{align}
    \mathbf{x}_k &= A_{k-1} \, \mathbf{x}_{k-1} + B_{k-1} \, a_{k-1} + \mathbf{w}_{k-1}, \quad \mathbf{w}_{k-1} \sim \mathcal{N}(0, Q_{k-1}) \\
    y_k &= r_k + v_k, \quad\quad\quad\quad\quad\quad\quad\quad\quad\quad\quad v_k \sim \mathcal{N}(0, R_k)
\end{align}

Here, \( \mathbf{x}_k \in \mathbb{R}^2 \) is the discrete-time state vector representing position and velocity,  
\( a_{k-1} \) is the sampled acceleration input, and  \( y_k \) is the noisy position measurement.
\( R_k \) is defined by the position sensor noise model.

The matrices \( A_{k-1} \), \( B_{k-1} \) and \( Q_{k-1} \)  are derived by discretizing the continuous-time 
system in equation \(\ref{eq:continuous_system}\). The process is as follows:

\clearpage

\textbf{Step 1: Compute \( A_d \) and \( B_d \) using ZOH}

\begin{align}
    M_{\text{zoh}} &= \begin{bmatrix}
    A & B \\
    0 & 0
    \end{bmatrix} \cdot \Delta t \\
    \begin{bmatrix} A_d & B_d \end{bmatrix} &= \exp(M_{\text{zoh}})
\end{align}

The process noise covariance \( Q_{k-1} \) can be computed using Van Loan’s method.

\textbf{Step 2: Compute \( Q_k \) using Van Loan's method}

\begin{align}
    M_{\text{vl}} &= \begin{bmatrix}
    -A & (L Q_c L^T) \\
    0 & A^T
    \end{bmatrix} \cdot \Delta t \\
    \Phi &= \exp(M_{\text{vl}}) \\
    Q_k &= A_d \cdot \Phi_{12}
\end{align}