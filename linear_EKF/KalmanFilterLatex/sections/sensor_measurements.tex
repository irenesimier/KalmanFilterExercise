\section{Sensor Measurements}

In this system, we use two sensors to measure the position 
and acceleration of the mass.

The acceleration measurement, denoted \( u_{\text{acc}}(t) \), 
is given by

\begin{equation}
    \begin{aligned}
        u_{\text{acc}}(t) &= \ddot{r}(t) + w(t), 
        \quad\quad w(t) \sim \mathcal{N}(0, Q(t)) \\
        u_{\text{acc}}(t) &= \frac{1}{m} \left( f(t) - k r(t) - c \dot{r}(t) \right) + w(t), 
    \end{aligned}
    \label{eq:acc_measurement}
\end{equation}


where \( w(t) \) is zero-mean Gaussian noise with time-varying 
covariance \( Q(t) \).
\clearpage

The position sensor provides a noisy measurement of the true position:

\begin{equation}
    y(t) = r(t) + v(t),
    \quad\quad v(t) \sim \mathcal{N}(0, R(t))
    \label{eq:pos_measurement}
\end{equation}

where \( v(t) \) is zero-mean Gaussian noise with covariance \( R(t) \).

Together, these measurements are used for state estimation in 
filtering and control algorithms.